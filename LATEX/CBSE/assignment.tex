\documentclass{article}
\usepackage[left=0.75in,right=0.75in,top=0.75in,bottom=0.75in]{geometry}
\usepackage{tfrupee}  
\usepackage{enumitem}
\usepackage[cmex10]{amsmath}
\providecommand{\cbrak}[1]{\ensuremath{\left\{#1\right\}}}
\providecommand{\brak}[1]{\ensuremath{\left(#1\right)}}
\providecommand{\rbrak}[1]{\ensuremath{\left[#1\right]}}
\providecommand{\dec}[2]{\ensuremath{\overset{#1}{\underset{#2}{\gtrless}}}}
\newcommand{\myvec}[1]{\ensuremath{\begin{pmatrix}#1\end{pmatrix}}}
\newcommand{\myaugvec}[2]{\ensuremath{\begin{amatrix}{#1}#2\end{amatrix}}}
\newcommand{\mydet}[1]{\ensuremath{\begin{vmatrix}#1\end{vmatrix}}}
\providecommand{\mydet}[1]{\ensuremath{\begin{vmatrix}#1\end{vmatrix}}}
\providecommand{\myvec}[1]{\ensuremath{\begin{bmatrix}#1\end{bmatrix}}}
\title{\textbf{MATHEMATICS}}
\author{SECTION A}
\date{\today}
\begin{document}
\maketitle
\begin{enumerate}
\section{Vectors}
\item Show that the points $\textbf{A, B, C}$ with position vectors $2 \hat{i}-\hat{j}+\hat{k}$, $\hat{i}-3 \hat{j}-5 \hat{k}$ and $3 \hat{i}-4 \hat{j}-4 \hat{k}$ respectively, are the vertices of a right-angled triangle. Hence find the area of the triangle.
\item Find the coordinates of the point where the line through the points $\brak{3,-4,-5}$ and $\brak{2,-3,1}$, crosses the plane determined by the points $\brak{1,2,3},\brak{4,2,-3}$ and $\brak{0,4,3}$.
\section{Linear Forms}
\item The x-coordinate of a point on the line joining the points $P\brak{2,2,1}$ and $Q\brak{5,1,-2}$ is $4$ . Find its z-coordinate.
\item Find the value of $x$ such that the points $A\brak{3,2,1}, B\brak{4, x, 5}, C\brak{4,2,-2}$ and $D\brak{6,5,-1}$ are coplanar.
\item Find the distance between the planes $2x-y+2z=5$ and $5x-2.5y+5z=20$.
\section{Probability}
\item A die, whose faces are marked $1,2,3$ in red and $4, 5, 6$ in green, is tossed. Let $A$ be the event "number obtained is even" and $B$ be the event "number obtained is red". Find if $A$ and $B$ are independent events.
\item There are $4$ cards numbered $1$, $3$, $5$ and $7$, one number on one card. Two cards are drawn at random without replacement. Let $X$ denote the sum of the numbers on the two drawn cards. Find the mean and variance of $X$.
\item Of the students in a school, it is known that $30 \%$ have $100\%$ attendance and $70 \%$ students are irregular. Previous year results report that $70 \%$ of all students who have $100 \%$ attendance attain A grade and $10\%$ irregular students attain A grade in their annual examination. At the end of the year, one student is chosen at random from the school and he was found to have an A grade. What is the probability that the student has $100 \%$ attendance$?$ Is regularity required only in school$?$ Justify your answer.
\section{Optimization}
\item Two tailors, $A$ and $B$, earn \rupee$300$ and \rupee$400$ per day respectively. $A$ can stitch $6$ shirts and $4$ pairs of trousers while $B$ can stitch $10$ shirts and $4$ pairs of trousers per day. To find how many days should each of them work and if it is desired to produce at least $60$ shirts and $32$ pairs of trousers at a minimum labour cost, formulate this as an LPP.
\item Solve the following linear programming problem graphically :\newline Maximise $Z=34x+45y$ \newline under the following constraints
\begin{align*}
x+y \leq 300 \\
2 x+3 y \leq 70 \\
x \geq 0, y \geq 0
\end{align*}
\section{Geometry}
\item $AB$ is the diameter of a circle and $C$ is any point on the circle. Show that the area of triangle $ABC$ is maximum, when it is an isosceles triangle.
\item A variable plane which remains at a constant distance $3p$ from the origin cuts the coordinate axes at $A, B, C$. Show that the locus of the centroid of triangle $ABC$ is $\dfrac{1}{x^2}+\dfrac{1}{y^2}+\dfrac{1}{z^2}=\dfrac{1}{p^2}$.
\section{Differentiation}
\item Show that the function $f\brak{x}=x^3-3 x^2+6 x-100$ is increasing on $R$.
\item The length $x$, of a rectangle is decreasing at the rate of $5~cm$ minute and the width $y$, is increasing at the rate of $4~cm /$ minute. When $x=8~cm$ and $y=6~cm$, find the rate of change of the area of the rectangle.
\item Find the particular solution of the differential equation $\brak{x-y}\dfrac{dy}{dx}=\brak{x+2y}$, given that $y=0$ when $x=1$.
\item Find the general solution of the differential equation
\begin{align*}
  ydx-\brak{x+2 y^2}dy=0.  
\end{align*}
\item If $x^y+y^x=a^b$, then find $\dfrac{dy}{dx}$.
\item If $e^{y}\brak{x+1}=1$, then show that $\dfrac{d^2y}{dx^2}$=$\brak{\dfrac{dy}{dx}}^2$.
\section{Integration}
\item Find :\begin{align*} \int \dfrac{\sin ^2 x-\cos ^2 x}{\sin x \cos x}\,dx\end{align*}
\item Find :\begin{align*}\int\frac{dx}{5-8 x-x^2}\end{align*}
\item Evaluate :\begin{align*}\int_0^\pi \frac{x \tan x}{\sec x+\tan x} \, dx\end{align*}
\item Evaluate :\begin{align*}\int_1^4\cbrak{|x-1|+|x-2|+|x-4|}\, dx\end{align*}
\item Using the method of integration, find the area of the triangle $ABC$, coordinates of whose vertices are $A\brak{4,1},B\brak{6,6}$ and $C\brak{8,4}$.
\item Find the area enclosed between the parabola $4y=3x^2$ and line $3x-2y+12=0$.
\item Find :\begin{align*}\int\dfrac{\sin\theta ~d\theta}{\brak{4+\cos^{2}\theta}\brak{2-\sin^{2}\theta}}\end{align*}
\section{Functions}
\item Find the value of $c$ in Rolle's theorem for the function $f\brak{x}=x^3-3 x$ in $\rbrak{-\sqrt{3}, 0}$.
\item Consider $f:R-\cbrak{-\dfrac{4}{3}}\rightarrow R-\cbrak{\dfrac{4}{3}}$ given by $f\brak{x}=\dfrac{4x+3}{3x+4}$. Show that $f$ is bijective. Find the inverse of $f$ and hence find $f^{-1}\brak{0}$ and $x$ such that $\mathrm{f}^{-1}\brak{x}=2$.
\item Let $A =Q \times Q$ and let $\ast$ be  a binary operation on $A$ defined by $\brak{a, b}$  $\ast$  $\brak{c, d}$ = $\brak{{a}{c}, b + {a}{d}}$ for $\brak{a, b}, \brak{c, d} \in A$. Determine, whether $\ast$ is commutative and associative. Then, with respect to $\ast$ on  $A$.
	\begin{enumerate}
		\item Find the identity element in $A$.
		\item Find the invertible elements of $A$.
	\end{enumerate}
\section{Matrices}
\item If for any $2\times2$ square matrix $A$, $A\brak{Adj A} =\myvec{8&0\\0&8}$,then write the value of $\det{A}$.
\item If $A$ is a skew-symmetric matrix of order $3$ , then prove that $\mydet{A}=0$.
\item Using properties of determinants, prove that \begin{align*}\mydet{a^{2}+2a & 2a+1 & 1 \\ 2a+1 & a+2 & 1 \\ 3 & 3 & 1} = \brak{a-1}^3\end{align*}
\item Find matrix $A$ such that
\begin{align*}
   \myvec{2 & -1 \\1 & 0 \\-3 & 4} A=\myvec{-1 & -8 \\1 & -2 \\9 & 22} 
\end{align*}
\item If $A=\myvec{2 & -3 & 5\\3 & 2 & -4\\1 & 1 & -2}$, find $A^{-1}$. Hence using $A^{-1}$ solve the system of equations \begin{align*}
    2x-3y+5z=11\\3x+2y-4z=-5\\x+y-2z=-3
\end{align*}
\section{Trigonometry}
\item If $\tan ^{-1} \dfrac{x-3}{x-4}+\tan ^{-1} \dfrac{x+3}{x+4}=\dfrac{\pi}{4}$, then find the value of $x$.
\section{Limits and Continuity}
\item Determine the value of '$k$' for which the following function is continuous at $x=3$ :
\begin{align*}
  f\brak{x}=\left\{\begin{array}{cc}
\dfrac{\brak{x+3}^2-36}{x-3} & , x \neq 3 \\
k & , x=3
\end{array}\right.  
\end{align*}
\end{enumerate}
\end{document}
